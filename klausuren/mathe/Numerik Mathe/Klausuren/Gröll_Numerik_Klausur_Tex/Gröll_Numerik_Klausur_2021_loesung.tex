\documentclass[12pt]{article}
\usepackage{tikz}
\usepackage{amssymb}
\usepackage{amsmath}
\usepackage{mathtools}
\usepackage{amsthm}
\usetikzlibrary{arrows,automata,positioning}
\usepackage{enumitem}
\usepackage{todonotes}
\usepackage{fancyhdr}
\usepackage{caption}
\captionsetup[table]{name=Tabelle}
\usepackage[ngerman]{babel}
\usepackage[utf8x]{inputenc}

\usepackage{geometry}
\geometry{
 a4paper,
 total={170mm,257mm},
 left=25mm,
 top=35mm,
 bottom=25mm,
 right=25mm
}
\newcommand{\rom}[1]{\uppercase\expandafter{\romannumeral #1\relax}}
\newcommand{\padTable}[1]{$\quad$ #1 $\quad$}
\newcommand{\ankerAuf}{⊣} % nur mit XeLaTeX, da Unicode!
\newcommand{\ankerZu}{⊢}
\pagestyle{fancy}
%\DeclareUnicodeCharacter{}{\ankerZu}
\fancyhf{}
\rhead{Klausur Sose 2021}
\lhead{Mathematik \rom{4} $-$ Numerik }
\cfoot{Seite \thepage}
\lfoot{}


\usepackage{xcolor}
\usepackage{environ}
\usepackage[most]{tcolorbox}
\tcbset{on line,
        boxsep=4pt, left=0pt,right=0pt,top=0pt,bottom=0pt,
        colframe=white,colback=lightgray,  
        highlight math style={enhanced}
        }
\newcommand{\solutionMacro}[1]{\,\\\texttt{#1}\,\\}
\NewEnviron{solution}{\textcolor{blue}{\solutionMacro{\BODY}}}

\renewcommand{\footrulewidth}{1pt}
\title{Mathematik \rom{4} \\ Numerik \\ Lutz Gröll $-$ Klausur  SoSe 2021}
\author{TINF19B2 $-$ Viel größere Fans des Gröll!}

\begin{document}
\maketitle

\begin{center}
	\large
	\vspace*{8cm}
 	Maximale Punktzahl: 59 Punkte\\
	\vspace*{1cm}
	Bearbeitungszeit: 90 Minuten\\
	\vspace*{1cm}
	Hilfsmittel: Taschenrechner + Formelblatt (für LA und Analysis, siehe Ordner)\\
	\vspace*{1cm}
	Datum: 09.06.2021\\
	\vspace*{1cm}
	\small In korrektem Wortlaut rekonstruiert - Satzzeichen teilweise korrigiert.\\Format der folgenden Seiten ist dem der Klausur sehr ähnlich.\\
\end{center}
\pagebreak



\section*{Aufgabe 1: (9 Punkte)}

\begin{enumerate}
	\item Notieren Sie die wichtigsten Schritte für das Erstellen eines numerischen Programms.
	
	\begin{solution}
	\begin{enumerate}[label=$(\roman*)$]
		\item Geeigneten Algorithmus wählen
		\item Ein- und Ausgabevariablen festlegen (und Datentypen etc.)
		\item Kommentare schreiben (Funktion des Programms, \\Spezifikation der Schnittstelle und Quelle)
		\item Interne Variablen festlegen
		\item Eigentliches Programm schreiben
		\item Testen
		\item Optional: Optimieren
		\item Numerische Bugs suchen und beseitigen (Singuläre Matrix?,\\ Nahezu singulär? Negativer Radikant? etc.)
		\item Außerdem: Echtzeitanforderungen? Sicherheitsaspekte beachtet?
	\end{enumerate}
	\end{solution}
	\vspace{2cm}
	
	\item Nennen Sie 4 Verfahren zur numerischen Lösung eines Problems und stellen Sie das zugehörige analytische Problem (Beispiel) gegenüber.

	\begin{solution}
	\begin{tabular}{l|l}
		Numerisches Verfahren: & Analytisches Problem:\\
		Runge-Kutta-4-Verfahren & Lösen von DGLs \\
		Trapezregel	& Berechnung eines bestimmten Integrals \\
		Newton-Raphson-Verfahren & Nullstellensuche \\
		Newton-Verfahren & Nichtlineare Optimierung
	\end{tabular}
	\end{solution}
\end{enumerate}

\pagebreak
\newcommand{\zeilensummennorm}[1]{||#1||_{\infty}}
\section*{Aufgabe 2: (11 Punkte)}
\begin{enumerate}
	\item Welchen Wert hat die Konditionszahl von $A=\begin{bmatrix}
	1 & 0 \\ 3 & 4
	 \end{bmatrix}$ in der Zeilensummennorm?
	 \begin{solution}
	$\kappa_{\infty}(A) = \zeilensummennorm{A}\cdot\zeilensummennorm{A^{-1}} = 7 \cdot \frac{1}{4} \cdot 4$
	\end{solution}
	
	\item Formulieren Sie die Berechnung von $x = B^{-1}Cd$ in eine numerisch effiziente Form um.
	\begin{solution}
		$Bx = Cd\qquad $
	\end{solution}
	
	\item Zeigen Sie an einem Beispiel, dass die Addition numerisch nicht assoziativ ist.
	
\newcommand{\numPlus}{\,\widetilde{+}\,}
	\begin{solution}
		$(10^{-20} \numPlus 1) \numPlus (-1) \neq 10^{-20} \numPlus (1 \numPlus (-1))$
	\end{solution}
	
	\item Nennen Sie die 3 Bedingungen für ein well-posed Problem.
	
	\begin{solution}
	Existenz einer Lösung, Eindeutigkeit der Lösung und stetige Abhängigkeit von den Eingangsdaten (Hadamard-Bedingungen).
	\end{solution}
	
	\item Notieren Sie ein Least-Squares-Problem mit Tikhonov Regularisierung.\\
		\begin{solution}
			$||Ax - b||_2^2 + \mu \cdot ||x||_2^2 \rightarrow Min$\\
			LS-Lösung (ineffizient): $x = (A^TA + \mu\cdot I_n)^{-1} A^Tb$
		\end{solution}

	
	\item Warum kann es beim Lösen der Differentialgleichung $\dot{x_1} = x_2 - k \sqrt{x_1}$ mit \mbox{$x\geq 0$} sinnvoll sein, eine Modifikation des Vektorfelds vorzunehmen? Welche Lösung schlagen Sie vor?
	\begin{solution}
		Nicht sicher, ob das die korrekte Lösung ist, aber es kann durch numerische Fehler passieren, dass plötzlich $x_1 < 0$, sodass der Radikant negativ wird.\\
		Meine Lösung: $\dot{x_1} = x_2 - k\sqrt{max\{x_1, 0\}}$
	\end{solution}

	
	\item Ein Algorithmus hat die Komplexität $\mathcal{O}(n^2)$. Heißt das, a) dass er weniger Aufwand als $n^2$ Operationen benötigt, b) mindestens $n^2$ Operationen benötigt, c) genau $kn^2$ mit $k\in \mathbb{N}$ Operationen benötigt oder ist d) keine der Aussagen richtig?
	\begin{solution}
		d) ist korrekt.	
	\end{solution}

	\pagebreak

\pagebreak
	\item Wodurch sind Testmatrizen für numerische Leistungstests gekennzeichnet?
\vspace{-0.5cm}	
	\begin{solution}
		\begin{enumerate}[label=$(\roman*)$]
			\item In der Ordnung skalierbar (zum Beispiel zum Testen des Speicherverbrauchs)
			\item Lösung bekannt
			\item Testen numerische Grenzfälle aus (Singularitäten, nahezu singulär etc.)
			\item In der Kondition veränderlich
			\item Stehen in Büchern!
		\end{enumerate}			
	\end{solution}



\end{enumerate}

\pagebreak

\section*{Aufgabe 3: (9 Punkte)}

\begin{enumerate}

	\item Nennen Sie eine praktische Anwendung, für die eine Interpolation nach Lagrange in Frage kommt.
	\begin{solution}
		Bei Hashtabellen zum Berechnen eines Zwischenwertes (falls die\\ eigentliche Funktion zu aufwendig ist oder eine Approximation reicht).
	\end{solution}

	
	
	\item Notieren Sie für $y = \frac{ax + b}{x^2 + cx + d}$ einen linearen LS-Ansatz.
	\begin{solution}
	Ausmultiplizieren gibt $yx^2 + cyx + yd - ax - b = 0$. Daraus dann den LS-Ansatz:\\
	
	$Ax = 0_n,\qquad A = \begin{bmatrix}
		y_1x_1^2 & y_1x_1 & y_1 & -x_1 & -1\\
		\vdots & \vdots & \vdots & \vdots & \vdots\\
		y_nx_n^2 & y_nx_n & y_n & -x_n & -1
	\end{bmatrix}, \qquad x = \begin{bmatrix}
		1\\c\\d\\a\\b
	\end{bmatrix}$.
	\end{solution}
	
	\item Wie viele Stützwerte benötigen Sie mindestens, um die Parameter aus Teilaufgabe 2 eindeutig bestimmen zu können?
	\begin{solution}
		4 Stützwerte
	\end{solution}

	
	
	\item Wie viele Funktionsaufrufe benötigen Sie mindestens für die numerische Approximation einer dritten Ableitung?
	\begin{solution}
		Für die $n=3$ Ableitung benötigt man $n+1$ Stützwerte.\\Hier also 4 Funktionsaufrufe.
	\end{solution}

	
	\item In welchem Konflikt stehen Ingenieure, die online eine Ableitung berechnen müssen?
	\begin{solution}
		Sie kennen die zukünftigen Werte nicht und können um den neusten Wert deshalb niemals symmetrisch differenzieren.	
	\end{solution}

	
	\item Was halten Sie von $f_k'' = -\frac{1}{12}f_{k-3}+ \frac{1}{3}f_{k-2} + \frac{1}{2}f_{k-1} - \frac{5}{3}f_k + f_{k+1}$?
	\begin{solution}
		Gar nichts. Die Summe der Gewichte müssen beim Ableiten 0 sein (wegen $f(x) = c, c\in \mathbb{R}$. Außerdem muss man noch durch $h^2$ für die zweite Ableitung teilen.	
	\end{solution}

\pagebreak

	\item Kann für $x^3(x-1) = 1$ der exakte Wert für $x\approx -0.8$ durch die Fixpunktiteration $x_{k+1} = \sqrt[3]{\frac{1}{x_k - 1}}$ berechnet werden? Führen Sie hierzu eine Konvergenzbetrachtung durch.\\
	\begin{solution}
		$\Phi(x_k) = \sqrt[3]{\frac{1}{x_k-1}},\quad \Phi'(x_k) = -\frac{1}{3}\cdot (x_k - 1)^{-\frac{4}{3}}$ Tatsächlich $||\Phi'(-0.8)|| < 1$.\\
		Deshalb ja, denn der Fixpunkt ist attraktiv.
	\end{solution}

	
	\item Warum werden Eigenwerte von Matrizen numerisch nicht wie in der Algebra üblich über die charakteristische Gleichug bestimmt? Was macht man stattdessen?\\
	\begin{solution}
	Da Polynome numerisch unvorteilhaft sind (Überläufe, aufwendige\\Berechnung und aufwendige Nullstellensuche). Bei $100\times 100$ Matrix ein\\Polynom 100. Grades. Stattdessen reelle QZ-Zerlegung \\(durch Gram-Schmidt-Orthogonalisierung).
	\end{solution}

\end{enumerate}




\pagebreak

\newcommand{\realVector}[2]{#1\in \mathbb{R}^{#2}}
\newcommand{\realMatrix}[3]{#1\in \mathbb{R}^{#2\times #3}}
\newcommand{\indentTab}{\hphantom{~~~~}}

\section*{Aufgabe 4: (9 Punkte)}

\begin{enumerate}

	\item Gegeben seien $\realMatrix{A}{30}{10}, \realMatrix{B}{10}{100}, \realVector{C}{100}$. Berechnen Sie die Flops für $A(BC)$.
	\begin{solution}
		$D = BC, \realVector{D}{10},\qquad Flop(A(BC)) = \underbrace{Flop(AD)}_{30\cdot 1\cdot (2\cdot 10 - 1) = 570} + \underbrace{Flop(BC)}_{10\cdot 1\cdot (2\cdot 100 - 1) = 1990} = 2560$
	\end{solution}
	
	\item Mit welchem Algorithmus können Sie die Funktionsaufrufe für eine rationale Funktion reduzieren?
	\begin{solution}
		Über Kettenbrüche (konnte sich niemand mehr aktiv dran erinnern).
	\end{solution}
	
	\item Welche Vorraussetzung muss für eine Parallelisierung eines Programms vorliegen? Nennen Sie ein Beispiel, wo Prallelisierung auf 8 Rechnerkernen leicht anwendbar ist und viel bringt.
	\begin{solution}
	Unabhängige Teilaufgaben. Beispiel: Matrixmultiplikation mit Matrizen $\realMatrix{A}{8}{8}, \realMatrix{B}{8}{8}, A\cdot B = C$
	\end{solution}
	
	\item Schreiben Sie in Pseudocode einen Test, um numerische Bugs bei der Auswertung von $tan\,x$ zu verhindern.
	\begin{solution}
		double ownTangent(double x)\\
			\indentTab x' := (x mod $\pi$) - $\frac{\pi}{2}$\\
			\indentTab if ($abs(x') \leq eps$) throw error;\\
			\indentTab else return tan(x);\\
	\end{solution}
	
	\item Was verstehen Sie unter Pivotisierung?\\
	Erklären Sie, worin der Nutzen dieser Technik liegt.

	\begin{solution}
		Das Auswählen eines Elementes nach einer definierten Strategie.\\
		Beim Gauß-Algorithmus sucht man das betragsmäßig größte Element der\\ Spalte/Zeile oder Matrix und nimmt dieses als Pivotelement. Numerisch ist dies überlegen, da numerische Fehler minimiert werden.
	\end{solution}	
	
	\item Weisen Sie nach, dass Matrizen vom Typ $A = \begin{bmatrix} \alpha & \beta \\ -\beta & \alpha \end{bmatrix}$ die  Möglichkeit bieten, konjugierte Eigenwerte in reeller Form darzustellen.
	
	\begin{solution}
		Charakteristisches Polynom: $det(A-\lambda I_n) = (\alpha - \lambda)^2 +\beta^2 \stackrel{!}{=} 0$.\\
		Offensichtlich gilt damit $\alpha-\lambda = \pm i\cdot \beta \iff \lambda_{1/2} = \alpha \pm i\cdot \beta$.
		QED.		
	\end{solution}
	
\pagebreak

	\item Bestimmen Sie ein $\epsilon$, bis zu dem Sie sich $x=1$ nähern können, ohne dass die Kondition von $f(x) = \frac{1}{(x-1)^2}$ den Wert $\kappa = 10^6$ übersteigt.
	
	\begin{solution}
	
	$\kappa_{rel} = 10^6 < | f'(x) \cdot \frac{x}{f(x)} |,\quad f'(x) = -2\cdot (x-1)^{-3}$\\
	$\implies 10^6 < 2\cdot |(x-1)^{-3} \cdot (x-1)^2 \cdot x| = 2\cdot 
	|\frac{x}{x-1}| |_{x=1\pm \epsilon}$\\
	$\implies 10^6 < 2\cdot|\frac{1\pm \epsilon}{1\pm \epsilon - 1}| \approx \frac{2}{\epsilon}$.
	
	Also $\epsilon < 2 \cdot 10^{-6}$.
	\end{solution}
\end{enumerate}

\pagebreak

\newcommand{\raphsonFunction}{
	$f(x_1,x_2) = 
		\begin{bmatrix}
			x_1^2 + x_2\\ x_1x_2 + x_2^2
		\end{bmatrix}$}

\newcommand{\raphsonStartVal}{
	$\begin{bmatrix}x_{10} \\ x_{20}\end{bmatrix}
	= 
	\begin{bmatrix}1\\ -1\end{bmatrix}$}

\section*{Aufgabe 5: (9 Punkte)}
\begin{enumerate}

	\item Leiten Sie das Newton-Verfahren zur Lösung von Optimierungsaufgaben her und geben Sie die recheneffiziente Version an.
	\begin{solution}
		$g_k$ := Gradient an Stelle $x_k$, $H_k$ := Hesse-Matrix an Stelle $x_k$.\\
		$Q(x) = Q(x_k) + g_k^T\cdot (x-x_k) + \frac{1}{2} (x-x_k)\cdot H_k \cdot (x-x_k) + Rest$\\\\
		Quadratisches Ersatzproblem:\\
		$Q(x) = Q(x_k) + g_k^T\cdot (x-x_k) + \frac{1}{2} (x-x_k)\cdot H_k \cdot (x-x_k)$\\
		$\frac{\delta Q}{\delta x} \stackrel{!}{=} 0 \iff g_k + H_k\cdot (x-x_k)|_{x=x_{opt}} = 0 \implies x_{k+1} := x_{opt} = x_k - H_k^{-1}\cdot g_k$\\\\
		Effiziente Variante:\\
		$x_{k+1} = x_k - \Delta x_k, \qquad H_k\cdot \Delta x_k = g_k$.
	\end{solution}
	
	
	\item Erklären Sie das Prinzip der Aktiven Mengenstrategie in der Optimierung.
	\begin{solution}
		``Entlanghangeln`` an aktiven Constraints (Ungleichungen, die mit $=$\\ erfüllt werden) bis zum Minimierer oder z. B. bis der Gradient beim\\ Gradientenabstiegsverfahren in einer andere Richtung zeigt.
	\end{solution}
	
	\item Definieren Sie superlineare Konvergenz.
	\begin{solution}
		$||x_{k+1} - x^{*}|| \leq c_k \cdot ||x_k - x*||^p, \quad c_k \rightarrow 0$ für $k \rightarrow \infty$\\
		Kontraktionskonstante $c_k$ geht mit steigender Iteration gegen 0.
	\end{solution}
	
	\item Warum ist das Newton-Verfahren zur Lösung von Aufgaben $c^T x \rightarrow Min$ unter $Ax = b$ und $Cx \leq d$ nicht geeignet?
	
	\begin{solution}
		Weil $c^T x$ linear ist und somit die Hesse-Matrix der Nullmatrix entspricht und somit nicht invertierbar ist.
	\end{solution}

	\item Wie viele zweite Ableitungen benötigen Sie beim Newton-Verfahren bei einem $p$-parametrischen Problem?
	
	\begin{solution}
		$\frac{p\cdot(p+1)}{2}$ Ableitungen, solange Satz von Schwartz erfüllt ist (also gdw. stetige differenzierbar bzw. wie Gröll es schreiben würde $f\in \mathcal{C}^2$ (zweimal stetig differenzierbar).
	\end{solution}	
	
\pagebreak
	\item Berechnen Sie den ersten Schritt der Newton-Raphson-Iteration zur Nullstellensuche von \raphsonFunction, wenn Sie mit \raphsonStartVal\, starten.
	\begin{solution}
		$f(\begin{bmatrix}x_{10} \\ x_{20}\end{bmatrix}) = \begin{bmatrix}0 \\ 0\end{bmatrix}$.\\
		Also effektiv ist der Startwert bereits die Lösung.\\
		Beachte: Die Jacobi-Matrix ist singulär und damit nicht invertierbar. Du könntest also hier keine ohnehin keine Iteration durchrechnen.\\
		Gröll meinte nach der Klausur, das sei ihm gar nicht aufgefallen,\\da er immer nur an ein paar Zahlen aus Altklausuren rumdreht.
	\end{solution}	
	

\end{enumerate}

\pagebreak
\section*{Aufgabe 6: (12 Punkte)}

\begin{enumerate}
	\item Formen Sie die Differenzialgleichung $y''' + x^2y = 1$ so um, dass Sie sie mit dem Runge-Kutta-Verfahren integrieren könnten.
	\begin{solution}
	Substitutionen: $y_1 = y,\quad y_2 = y',\quad y_3 = y''$.\\\\
	DGL-System:
	\begin{enumerate}
		\item $y_1' = y_2$
		\item $y_2' = y_3$
		\item $y_3' = -x^2y_1 + 1$
	\end{enumerate}
	
	Anfangswerte nicht vergessen! (Fände Gröll gar nicht lustig!)\\
	$y_1(x_{10}) = y_0,\quad y_2(x_{20}) = y_0', \quad y_3(x_{30}) = y_0''$
	\end{solution}
	\item Notieren Sie für ein Cauchy-Problem eine Funktionsdefinition für das Lösen eines $p$-dimensionalen Differentialgleichungssystems erster Ordnung.

	\begin{solution}
		[double[p, N], time[1,N]] ode(handle f, double[p] initvalues, double xstart,\\
		\indentTab \indentTab double xfinal, optional tol, optional verfahren)
	\end{solution}	
	
	
	\item Berechnen Sie den Wert $y(\frac{3}{2})$ der Differentialgleichung $y' = xy + x$ mit dem Runge-Kutta-4-Verfahren, wenn Ihr Anfangswert $y(1) = 2$ ist. Wählen Sie die Schrittweite $h = \frac{1}{2}$.
	\begin{solution}
	$y_1 = y_0 + \frac{h}{6} \cdot ( k_1 + 2k_2 + 2k_3 + k_4 )$\\
	\begin{enumerate}[label=$\,$]
		\item $k_1 = f(x_0, y_0)$
		\item $k_2 = f(x_0 + \frac{h}{2}, y_0 + \frac{h}{2}k_1)$
		\item $k_3 = f(x_0 + \frac{h}{2}, y_0 + \frac{h}{2}k_2)$
		\item $k_4 = f(x_0 + h, y_0 + h\cdot k_3)$
	\end{enumerate}
	Das ist mir zu blöd. Rechnet das doch selbst.
	\end{solution}

	
	\item Lösen Sie $Q = \int\limits_0^1 (x-1)^3 dx$ analytisch. Anschließend lösen Sie das Problem mit der Trapezregel numerisch. Verwenden Sie die Schrittweite $h = \frac{1}{4}$.
	\begin{solution}
	$f(x) = (x-1)^3$\\
	Analytisch $[\frac{1}{4}(x-1)^4]_0^1 = -\frac{1}{4}$.\\
	Trapez: $h \cdot (\frac{1}{2} f(0) + f(\frac{1}{4}) +f(\frac{1}{2})+f(\frac{3}{4})+\frac{1}{2} f(1)) \approx -0.26$ (glaub ich; Erneut: Mach selbst!)
	\end{solution}
\end{enumerate}


\end{document}
\documentclass[12pt]{article}
\usepackage{tikz}
\usepackage{graphicx}
\usepackage{amssymb}
\usepackage{amsmath}
\usepackage{mathtools}
\usepackage{amsthm}
\usetikzlibrary{arrows,automata,positioning}
\usepackage{enumitem}
\usepackage{todonotes}
\usepackage{fancyhdr}
\usepackage{wrapfig}
\usepackage{caption}
\captionsetup[table]{name=Tabelle}
\usepackage[ngerman]{babel}
\usepackage[utf8x]{inputenc}

\usepackage{geometry}
\geometry{
 a4paper,
 total={170mm,257mm},
 left=25mm,
 top=35mm,
 bottom=25mm,
 right=25mm
}
\newcommand{\rom}[1]{\uppercase\expandafter{\romannumeral #1\relax}}
\newcommand{\padTable}[1]{$\quad$ #1 $\quad$}
\newcommand{\ankerAuf}{⊣} % nur mit XeLaTeX, da Unicode!
\newcommand{\ankerZu}{⊢}
\pagestyle{fancy}
%\DeclareUnicodeCharacter{}{\ankerZu}
\fancyhf{}
\rhead{Klausur WiSe 2022}
\lhead{Mathematik \rom{3} $-$ Numerik}
\cfoot{Seite \thepage}
\lfoot{}


\usepackage{xcolor}
\usepackage{environ}
\usepackage[most]{tcolorbox}
\tcbset{on line,
        boxsep=4pt, left=0pt,right=0pt,top=0pt,bottom=0pt,
        colframe=white,colback=lightgray,
        highlight math style={enhanced}
        }
\newcommand{\kommentarMacro}[1]{\textcolor{lightgray}{(\texttt{#1})}}
\newcommand{\solutionMacro}[1]{\,\\\texttt{#1}\,\\}
\NewEnviron{solution}{\textcolor{blue}{\solutionMacro{\BODY}}}

\renewcommand{\footrulewidth}{1pt}
\title{Mathematik \rom{3} \\ Numerik \\ Lutz Gröll $-$ Klausur WiSe 2022}
\author{TINF21B1 \\ Lutz Gröll, Connaisseur der imaginären Schwarzwälder Kirschtorte}

\newsavebox\MBox
\newcommand\Cline[2][red]{{\sbox\MBox{$#2$}%
  \rlap{\usebox\MBox}\color{#1}\rule[-1.2\dp\MBox]{\wd\MBox}{0.5pt}}}


\begin{document}
\maketitle

\begin{center}
	\large
	\vspace*{7cm}
	Maximale Punktzahl: 69 Punkte\\
	\vspace*{1cm}
	Bearbeitungszeit: 90 Minuten\\
	\vspace*{1cm}
	Hilfsmittel: Taschenrechner + Formelblatt (für LA und Analysis, siehe Ordner)\\
	\vspace*{1cm}
	Datum: 06.12.2022\\
	\vspace*{1cm}
	\small \LaTeX-Dokument sowie teilweise Lösungen von 2021 übernommen \\Format der folgenden Seiten ist dem der Klausur sehr ähnlich.\\
\end{center}
\pagebreak



\section*{Aufgabe 1:}
\begin{enumerate}
	\item Notieren Sie die wichtigsten Schritte für das Erstellen eines numerischen Programms.

	      \vspace{9cm}

	\item Notieren Sie ein zweidimensionales Nullstellenproblem mit exakt 4 Lösungen (Formel und Skizze) \kommentarMacro{kein Originaltext}

	      \vspace{4cm}

	\item Nennen Sie ein ill-posed Problem für eine eindimensionale numerische Integration. \kommentarMacro{kein Originaltext}

\end{enumerate}
\pagebreak

\newcommand{\spaltensummennorm}[1]{||#1||_{1}}
\newcommand{\zeilensummennorm}[1]{||#1||_{\infty}}
\newcommand{\realVector}[2]{#1\in \mathbb{R}^{#2}}
\newcommand{\realMatrix}[3]{#1\in \mathbb{R}^{#2\times #3}}
\newcommand{\indentTab}{\hphantom{~~~~}}

\section*{Aufgabe 2:}
\begin{enumerate}
	\item Geben Sie eine $2\times2$ Matrix an, welche verschiedene Konditionszahlen in der Spaltensummennorm und in der Zeilensummennorm hat. \kommentarMacro{kein Originaltext}

	      \vspace{1.25cm}

	\item Formulieren Sie die Berechnung von $x = B^{-1}Cd$ in eine numerisch effiziente Form um.

	      \vspace{1.25cm}

	\item Zeigen Sie an einem Beispiel, dass die Addition numerisch nicht assoziativ ist.

	      \vspace{1.25cm}

	\item Was verstehen Sie unter Overfitting? \kommentarMacro{kein Originaltext}

	      \vspace{1.25cm}

	\item Was ist Spaltenskalierung? (Formel und Zweck/Anwendung) \\ \kommentarMacro{kein Originaltext}

	      \vspace{1.25cm}

	\item Sie haben ein schlecht konditioniertes Ausgleichsproblem aus dem ein Signal entstehen soll. Sie wollen dabei die 2. Ableitung bestrafen. Notieren Sie die Least-Square Formel, sowie die Matrix zur Bestrafung der 2. Ableitung. \\ \kommentarMacro{kein Originaltext}

	      \vspace{1.25cm}

	\item Warum kann es beim Lösen der Differentialgleichung $\dot{x_1} = x_2 - k \sqrt{x_1}$ mit \mbox{$x\geq 0$} sinnvoll sein, eine Modifikation des Vektorfelds vorzunehmen? Welche Lösung schlagen Sie vor?

	      \vspace{1.25cm}

	\item Wodurch sind Testmatrizen für numerische Leistungstests gekennzeichnet?

\end{enumerate}
\pagebreak

\section*{Aufgabe 3:}
\begin{enumerate}
	\item Nennen Sie eine praktische Anwendung, für die eine Interpolation nach Lagrange in Frage kommt.

	      \vspace{1.5cm}

	\item Notieren Sie für $y = \frac{ax + b}{x^2 + cx + d}$ einen linearen LS-Ansatz.

	      \vspace{1.5cm}

	\item Wie viele Stützwerte benötigen Sie mindestens, um die Parameter aus Teilaufgabe 2 eindeutig bestimmen zu können?

	      \vspace{1.5cm}

	\item In welchem Konflikt stehen Ingenieure, die online eine Ableitung berechnen müssen?

	      \vspace{1.5cm}

	\item Was halten Sie von $f_k'' = -\frac{1}{12}f_{k-3}+ \frac{1}{3}f_{k-2} + \frac{1}{2}f_{k-1} - \frac{5}{3}f_k + f_{k+1}$?

	      \vspace{1.5cm}

	\item Erstellen Sie für die Formel $x^3\left(x-1\right)=1$ zwei verschiedene Fixpunktiterationen. \kommentarMacro{kein Originaltext}

	      \vspace{1.5cm}


	\item Warum werden Eigenwerte von Matrizen numerisch nicht wie in der Algebra üblich über die charakteristische Gleichug bestimmt? Was macht man stattdessen?

	      \vspace{1.5cm}

	\item Skizzieren Sie eine instabile Fixpunktiteration graphisch.

\end{enumerate}
\pagebreak

\section*{Aufgabe 4:}
\begin{enumerate}
	\item Gegeben seien $\realMatrix{A}{30}{5}, \realMatrix{B}{5}{100}, \realVector{C}{100}$. Berechnen Sie die Flops für $A(BC)$.

	      \vspace{1.25cm}

	\item Sie benutzen einen Microcontroller, der nur die 4 Grundrechenarten beherrscht. Damit müssen Sie öfters Polynome der Form $y=a_3x^3+a_2x^2+a_1x+a_0$ berechnen. Wie viele Flops werden bei einer einfachen Berechnung benötigt? Wie viele hingegen, wenn das Horner-Schema zur Berechnung verwendet wird? \\ \kommentarMacro{kein Originaltext}

	      \vspace{1.5cm}

	\item Welche Vorraussetzung muss für eine Parallelisierung eines Programms vorliegen? Nennen Sie ein Beispiel, wo Prallelisierung auf 8 Rechnerkernen leicht anwendbar ist und viel bringt.

	      \vspace{1.5cm}

	\item Schreiben Sie in Pseudocode einen Test, um numerische Bugs bei der Auswertung von $tan\,x$ zu verhindern.

	      \vspace{1.5cm}

	\item Verbessert Pivotisierung die Kondition?

	      \vspace{1.25cm}

	\item Wie schafft der Numeriker einen Spaltentausch im Gauß-Algorithmus durchzuführen, ohne eine Dummy-Variable zu verwenden? \kommentarMacro{kein Originaltext}

	      \vspace{1.25cm}

	\item Weisen Sie nach, dass Matrizen vom Typ $A = \begin{bmatrix} \alpha & \beta \\ -\beta & \alpha \end{bmatrix}$ die  Möglichkeit bieten, konjugierte Eigenwerte in reeller Form darzustellen.

\end{enumerate}
\pagebreak

\newcommand{\raphsonFunction}{
	$f(x_1,x_2) =
		\begin{bmatrix}
			x_1^2 - x_2 \\ x_1x_2^2 + x_2
		\end{bmatrix}$}

\newcommand{\raphsonStartVal}{
	$\begin{bmatrix}x_{10} \\ x_{20}\end{bmatrix}
		=
		\begin{bmatrix}1\\ -1\end{bmatrix}$}

\section*{Aufgabe 5: (9 Punkte)}
\begin{enumerate}
	\item Leiten Sie das Newton-Verfahren zur Lösung von Optimierungsaufgaben her und geben Sie die recheneffiziente Version an.

	      \vspace{4cm}

	\item Erklären Sie das Prinzip der Aktiven Mengenstrategie in der Optimierung.

	      \vspace{2cm}

	\item Definieren Sie superlineare Konvergenz.

	      \vspace{2cm}

	\item Warum kann der Gradient einer $p$-dimensionalen Funktion in $p+1$ Funktionsaufrufen berechnet werden?

	      \vspace{1.5cm}

	\item Wie viele zweite Ableitungen benötigen Sie beim Newton-Verfahren bei einem $p$-parametrischen Problem?

	      \vspace{1.5cm}

	\item Berechnen Sie den ersten Schritt der Newton-Raphson-Iteration zur Nullstellensuche von \raphsonFunction, wenn Sie mit \raphsonStartVal\, starten.

\end{enumerate}
\pagebreak

\section*{Aufgabe 6: (9 Punkte)}
\begin{enumerate}
	\item Formen Sie die Differenzialgleichung $y''' + x^2y = 1$ so um, dass Sie sie mit dem Runge-Kutta-Verfahren integrieren könnten.

	      \vspace{2.5cm}

	\item Notieren Sie die Funktionsdefinition für das Lösen eines p-dimensionalen Differentialgleichungssystems erster Ordnung mit variabler Schrittweite.

	      \vspace{2.5cm}

	\item Berechnen Sie den Wert $y(\frac{5}{2})$ der Differentialgleichung $y' = xy + x$ mit dem Runge-Kutta-4-Verfahren, wenn Ihr Anfangswert $y(2) = 0$ ist. Wählen Sie die Schrittweite $h = \frac{1}{2}$.

	      \vspace{7cm}

	\item Bestimmen Sie ein $\epsilon$, bis zu dem Sie sich $x=1$ nähern können, ohne dass die Kondition von $f(x) = \frac{1}{(x-1)^2}$ den Wert $\kappa = 10^6$ übersteigt.

\end{enumerate}
\pagebreak

\end{document}
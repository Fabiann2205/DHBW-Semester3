\documentclass[12pt]{article}
\usepackage{tikz}
\usepackage{amssymb}
\usepackage{amsmath}
\usepackage{mathtools}
\usepackage{amsthm}
\usetikzlibrary{arrows,automata,positioning}
\usepackage{enumitem}
\usepackage{todonotes}
\usepackage{fancyhdr}
\usepackage{caption}
\captionsetup[table]{name=Tabelle}
\usepackage[ngerman]{babel}
\usepackage[utf8x]{inputenc}

\usepackage{geometry}
\geometry{
 a4paper,
 total={170mm,257mm},
 left=25mm,
 top=35mm,
 bottom=25mm,
 right=25mm
}
\newcommand{\rom}[1]{\uppercase\expandafter{\romannumeral #1\relax}}
\newcommand{\padTable}[1]{$\quad$ #1 $\quad$}
\newcommand{\ankerAuf}{⊣} % nur mit XeLaTeX, da Unicode!
\newcommand{\ankerZu}{⊢}
\pagestyle{fancy}
%\DeclareUnicodeCharacter{}{\ankerZu}
\fancyhf{}
\rhead{Klausur Sose 2021}
\lhead{Mathematik \rom{4} $-$ Numerik }
\cfoot{Seite \thepage}
\lfoot{}


\renewcommand{\footrulewidth}{1pt}
\title{Mathematik \rom{4} \\ Numerik \\ Lutz Gröll $-$ Klausur  SoSe 2021}
\author{TINF19B2 $-$ Viel größere Fans des Gröll!}

\begin{document}
\maketitle

\begin{center}
	\large
	\vspace*{9cm}
 	Maximale Punktzahl: 59 Punkte\\
	\vspace*{1cm}
	Bearbeitungszeit: 90 Minuten\\
	\vspace*{1cm}
	Hilfsmittel: Taschenrechner + Formelblatt (siehe Ordner)\\
	\vspace*{1cm}
	Datum: 09.06.2021\\
	\vspace*{1cm}
	\small In korrektem Wortlaut rekonstruiert - Satzzeichen teilweise korrigiert.\\Format der folgenden Seiten ist dem der Klausur sehr ähnlich.\\
\end{center}
\pagebreak



\section*{Aufgabe 1: (9 Punkte)}

\begin{enumerate}
	\item Notieren Sie die wichtigsten Schritte für das Erstellen eines numerischen Programms.
	
	\vspace{9cm}
	
	\item Nennen Sie 4 Verfahren zur numerischen Lösung eines Problems und stellen Sie das zugehörige analytische Problem (Beispiel) gegenüber.

\end{enumerate}

\pagebreak

\section*{Aufgabe 2: (11 Punkte)}
\begin{enumerate}
	\item Welchen Wert hat die Konditionszahl von $A=\begin{bmatrix}
	1 & 0 \\ 3 & 4
	 \end{bmatrix}$ in der Zeilensummennorm?
	
	
	\vspace{1.5cm}
	
	\item Formulieren Sie die Berechnung von $x = B^{-1}Cd$ in eine numerisch effiziente Form um.
	
	\vspace{1.5cm}
	
	\item Zeigen Sie an einem Beispiel, dass die Addition numerisch nicht assoziativ ist.
	
	\vspace{1.5cm}
	
	\item Nennen Sie die 3 Bedingungen für ein well-posed Problem.
	
	\vspace{1.5cm}
	
	\item Notieren Sie ein Least-Squares-Problem mit Tikhonov Regularisierung.
	
	\vspace{1.5cm}
	
	\item Warum kann es beim Lösen der Differentialgleichung $\dot{x_1} = x_2 - k \sqrt{x_1}$ mit \mbox{$x\geq 0$} sinnvoll sein, eine Modifikation des Vektorfelds vorzunehmen? Welche Lösung schlagen Sie vor?
	
	\vspace{1cm}
	
	\item Ein Algorithmus hat die Komplexität $\mathcal{O}(n^2)$. Heißt das, a) dass er weniger Aufwand als $n^2$ Operationen benötigt, b) mindestens $n^2$ Operationen benötigt, c) genau $kn^2$ mit $k\in \mathbb{N}$ Operationen benötigt oder ist d) keine der Aussagen richtig?
	
	\vspace{1cm}
	

	\item Wodurch sind Testmatrizen für numerische Leistungstests gekennzeichnet?
	


\end{enumerate}

\pagebreak

\section*{Aufgabe 3: (9 Punkte)}

\begin{enumerate}

	\item Nennen Sie eine praktische Anwendung, für die eine Interpolation nach Lagrange in Frage kommt.
	
	\vspace{1cm}
	
	
	\item Notieren Sie für $y = \frac{ax + b}{x^2 + cx + d}$ einen linearen LS-Ansatz.
	
	\vspace{1cm}
	
	\item Wie viele Stützwerte benötigen Sie mindestens, um die Parameter aus Teilaufgabe 2 eindeutig bestimmen zu können?
	
	\vspace{2cm}
	
	
	\item Wie viele Funktionsaufrufe benötigen Sie mindestens für die numerische Approximation einer dritten Ableitung?
	
	\vspace{1.5cm}
	
	\item In welchem Konflikt stehen Ingenieure, die online eine Ableitung berechnen müssen?
	
	\vspace{1.5cm}
	
	\item Was halten Sie von $f_k'' = -\frac{1}{12}f_{k-3}+ \frac{1}{3}f_{k-2} + \frac{1}{2}f_{k-1} - \frac{5}{3}f_k + f_{k+1}$?
	
	\vspace{1.5cm}
	
	\item Kann für $x^3(x-1) = 1$ der exakte Wert für $x\approx -0.8$ durch die Fixpunktiteration $x_{k+1} = \sqrt[3]{\frac{1}{x_k - 1}}$ berechnet werden? Führen Sie hierzu eine Konvergenzbetrachtung durch.
	
	\vspace{3cm}
	

	\item Warum werden Eigenwerte von Matrizen numerisch nicht wie in der Algebra üblich über die charakteristische Gleichug bestimmt? Was macht man stattdessen?


\end{enumerate}




\pagebreak

\newcommand{\realVector}[2]{#1\in \mathbb{R}^{#2}}
\newcommand{\realMatrix}[3]{#1\in \mathbb{R}^{#2\times #3}}

\section*{Aufgabe 4: (9 Punkte)}

\begin{enumerate}

	\item Gegeben seien $\realMatrix{A}{30}{10}, \realMatrix{B}{10}{100}, \realVector{C}{100}$. Berechnen Sie die Flops für $A(BC)$.
	
	\vspace{1cm}
	
	\item Mit welchem Algorithmus können Sie die Funktionsaufrufe für eine rationale Funktion reduzieren?
	
	\vspace{1cm}
	
	\item Welche Vorraussetzung muss für eine Parallelisierung eines Programms vorliegen? Nennen Sie ein Beispiel, wo Prallelisierung auf 8 Rechnerkernen leicht anwendbar ist und viel bringt.
	
	\vspace{2cm}
	
	
	\item Schreiben Sie in Pseudocode einen Test, um numerische Bugs bei der Auswertung von $tan\,x$ zu verhindern.
	
	\vspace{2cm}
	
	\item Was verstehen Sie unter Pivotisierung?\\
	Erklären Sie, worin der Nutzen dieser Technik liegt.


	\vspace{2cm}
	
	\item Weisen Sie nach, dass Matrizen vom Typ $A = \begin{bmatrix} \alpha & \beta \\ -\beta & \alpha \end{bmatrix}$ die  Möglichkeit bieten, konjugierte Eigenwerte in reeller Form darzustellen.
	
	\vspace{2.5cm}
	
	\item Bestimmen Sie ein $\epsilon$, bis zu dem Sie sich $x=1$ nähern können, ohne dass die Kondition von $f(x) = \frac{1}{(x-1)^2}$ den Wert $\kappa = 10^6$ übersteigt.
	

\end{enumerate}

\pagebreak

\newcommand{\raphsonFunction}{
	$f(x_1,x_2) = 
		\begin{bmatrix}
			x_1^2 + x_2\\ x_1x_2 + x_2^2
		\end{bmatrix}$}

\newcommand{\raphsonStartVal}{
	$\begin{bmatrix}x_{10} \\ x_{20}\end{bmatrix}
	= 
	\begin{bmatrix}1\\ -1\end{bmatrix}$}

\section*{Aufgabe 5: (9 Punkte)}
\begin{enumerate}

	\item Leiten Sie das Newton-Verfahren zur Lösung von Optimierungsaufgaben her und geben Sie die recheneffiziente Version an.
	
	\vspace{4cm}
	
	\item Erklären Sie das Prinzip der Aktiven Mengenstrategie in der Optimierung.
	
	\vspace{2cm}
	
	\item Definieren Sie superlineare Konvergenz.
	
	\vspace{2cm}
	
	\item Warum ist das Newton-Verfahren zur Lösung von Aufgaben $c^T x \rightarrow Min$ unter $Ax = b$ und $Cx \leq d$ nicht geeignet?
	
	\vspace{1.5cm}
	
	\item Wie viele zweite Ableitungen benötigen Sie beim Newton-Verfahren bei einem $p$-parametrischen Problem?
	
	\vspace{1.5cm}
	
	\item Berechnen Sie den ersten Schritt der Newton-Raphson-Iteration zur Nullstellensuche von \raphsonFunction, wenn Sie mit \raphsonStartVal\, starten.


\end{enumerate}

\pagebreak
\section*{Aufgabe 6: (12 Punkte)}

\begin{enumerate}
	\item Formen Sie die Differenzialgleichung $y''' + x^2y = 1$ so um, dass Sie sie mit dem Runge-Kutta-Verfahren integrieren könnten.
	
	\vspace{2cm}
	
	\item Notieren Sie für ein Cauchy-Problem eine Funktionsdefinition für das Lösen eines $p$-dimensionalen Differentialgleichungssystems erster Ordnung.
	
	\vspace{2cm}
	
	\item Berechnen Sie den Wert $y(\frac{3}{2})$ der Differentialgleichung $y' = xy + x$ mit dem Runge-Kutta-4-Verfahren, wenn Ihr Anfangswert $y(1) = 2$ ist. Wählen Sie die Schrittweite $h = \frac{1}{2}$.
	
	\vspace{7cm}
	
	\item Lösen Sie $Q = \int\limits_0^1 (x-1)^3 dx$ analytisch. Anschließend lösen Sie das Problem mit der Trapezregel numerisch. Verwenden Sie die Schrittweite $h = \frac{1}{4}$.
\end{enumerate}


\end{document}